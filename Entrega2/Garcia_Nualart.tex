% Options for packages loaded elsewhere
\PassOptionsToPackage{unicode}{hyperref}
\PassOptionsToPackage{hyphens}{url}
%
\documentclass[
]{article}
\usepackage{amsmath,amssymb}
\usepackage{lmodern}
\usepackage{iftex}
\ifPDFTeX
  \usepackage[T1]{fontenc}
  \usepackage[utf8]{inputenc}
  \usepackage{textcomp} % provide euro and other symbols
\else % if luatex or xetex
  \usepackage{unicode-math}
  \defaultfontfeatures{Scale=MatchLowercase}
  \defaultfontfeatures[\rmfamily]{Ligatures=TeX,Scale=1}
\fi
% Use upquote if available, for straight quotes in verbatim environments
\IfFileExists{upquote.sty}{\usepackage{upquote}}{}
\IfFileExists{microtype.sty}{% use microtype if available
  \usepackage[]{microtype}
  \UseMicrotypeSet[protrusion]{basicmath} % disable protrusion for tt fonts
}{}
\makeatletter
\@ifundefined{KOMAClassName}{% if non-KOMA class
  \IfFileExists{parskip.sty}{%
    \usepackage{parskip}
  }{% else
    \setlength{\parindent}{0pt}
    \setlength{\parskip}{6pt plus 2pt minus 1pt}}
}{% if KOMA class
  \KOMAoptions{parskip=half}}
\makeatother
\usepackage{xcolor}
\usepackage[margin=1in]{geometry}
\usepackage{graphicx}
\makeatletter
\def\maxwidth{\ifdim\Gin@nat@width>\linewidth\linewidth\else\Gin@nat@width\fi}
\def\maxheight{\ifdim\Gin@nat@height>\textheight\textheight\else\Gin@nat@height\fi}
\makeatother
% Scale images if necessary, so that they will not overflow the page
% margins by default, and it is still possible to overwrite the defaults
% using explicit options in \includegraphics[width, height, ...]{}
\setkeys{Gin}{width=\maxwidth,height=\maxheight,keepaspectratio}
% Set default figure placement to htbp
\makeatletter
\def\fps@figure{htbp}
\makeatother
\setlength{\emergencystretch}{3em} % prevent overfull lines
\providecommand{\tightlist}{%
  \setlength{\itemsep}{0pt}\setlength{\parskip}{0pt}}
\setcounter{secnumdepth}{5}
\pagenumbering{gobble}
\usepackage{subfig}
\usepackage[labelsep=period]{caption}
\usepackage[labelfont=bf]{caption}
\renewcommand{\and}{\\}
\ifLuaTeX
  \usepackage{selnolig}  % disable illegal ligatures
\fi
\IfFileExists{bookmark.sty}{\usepackage{bookmark}}{\usepackage{hyperref}}
\IfFileExists{xurl.sty}{\usepackage{xurl}}{} % add URL line breaks if available
\urlstyle{same} % disable monospaced font for URLs
\hypersetup{
  pdftitle={Número de pasajeros en el Puerto de Barcelona (2012-2019)},
  pdfauthor={Garcia Sató, Pol; Nualart Sanz, Arnau},
  hidelinks,
  pdfcreator={LaTeX via pandoc}}

\title{Número de pasajeros en el Puerto de Barcelona (2012-2019)}
\usepackage{etoolbox}
\makeatletter
\providecommand{\subtitle}[1]{% add subtitle to \maketitle
  \apptocmd{\@title}{\par {\large #1 \par}}{}{}
}
\makeatother
\subtitle{Segona pràctica Anàlisi de Series Temporals}
\author{Garcia Sató, Pol \and Nualart Sanz, Arnau}
\date{30 diciembre, 2022}

\begin{document}
\maketitle

\renewcommand{\contentsname}{Índex}
\renewcommand{\figurename}{Figura}
\renewcommand{\tablename}{Taula}

\captionsetup{width=.75\textwidth}

\vspace{120mm}

\includegraphics[width=0.5\textwidth,height=\textheight]{C:/Users/polgsDesktop/Grau estadística UB/4rt de carrera/1er Quadrimestre/AnalisisSeriesTemporals/Entrega2/Imatges/ub.png}

\includegraphics[width=0.5\textwidth,height=\textheight]{C:/Users/polgsDesktop/Grau estadística UB/4rt de carrera/1er Quadrimestre/AnalisisSeriesTemporals/Entrega2/Imatges/upc.png}

\newpage

\begin{center}
El objetivo de este informe es ver el comportamiento del número de pasajeros en el puerto de Barcelona durante los años 2012 - 2019 y hacer una predicción para el período 2020 - 2022 si no se tuvieran en cuenta los efectos causados por la pandemia del SARS-CoV-2. 

Se observa que los datos presentan tendencia creciente y estacionalidad multiplicativa. Es decir, con el paso de los años la demanda fue aumentando. Además, hay un claro patrón estacional que muestra que en los meses de verano el número es superior a los meses de otoño e invierno. 

En cuanto a la predicción, la demanda sigue aumentando de manera multiplicativa y siguiendo el mismo patrón estacional. 
\end{center}

\pagebreak

\tableofcontents

\pagebreak

\hypertarget{introducciuxf3}{%
\section{Introducció}\label{introducciuxf3}}

\hypertarget{descripciuxf3n-del-entorno-y-el-problema---motivaciuxf3n-del-trabajo}{%
\subsection{Descripción del entorno y el problema - motivación del
trabajo}\label{descripciuxf3n-del-entorno-y-el-problema---motivaciuxf3n-del-trabajo}}

Desde el estallido de la pandemia del SARS-CoV-2 las preferencias de la
población respecto a sus destinos vacacionales parecen haber cambiado.
Actualmente, los viajeros están más interesados en lugares naturales, al
aire libre, debido a que en la época del confinamiento mucha parte de la
población se sintió agobiada dentro de sus hogares. Adicionalmente, los
españoles se han centrado en un turismo nacional debido a la situación
de crisis sanitaria que estamos viviendo. Así pues, el número de
pasajeros en el Puerto de Barcelona puede haber disminuido drásticamente
ya que los cruceros son espacios cerrados y suelen ser vacaciones
internacionales.

El objetivo principal de este informe es ver cuál era el comportamiento
del número de pasajeros en el Puerto de Barcelona, ya sean de cruceros
como de ferris, en los años anteriores a la pandemia. De esta manera se
podrá ver de manera correcta la tendencia creciente que se tenía antes
de la crisis sin tener en cuenta la posible caída debida a la pandemia.

Para la elección de los años analizados, se ha evitado estudiar aquellos
momentos temporales en los que había un factor condicionante. Por tanto,
los datos tomados dejan fuera la crisis económica del 2008 y la pandemia
del 2020. También tendremos en cuenta que a principios del 2012 hubo el
desastre de la naviera italiana Costa Concordia con 32 muertos, lo cual
hizo que hubiera un descenso significativo en el número de pasajeros en
los cruceros del Mediterráneo. Teniendo esto en cuenta, solo se han
seleccionado los datos mensuales pertenecientes al espacio temporal
comprendido entre 2012 y 2019, ambos períodos incluidos.

\medskip

\end{document}
